\documentclass[12pt, a4paper]{article}
\usepackage[russian]{babel}
\usepackage{fontspec}
\setsansfont{Calibri}
\setmonofont{Consolas}
\setmainfont[
    Ligatures=TeX,
    Extension=.otf,
    BoldFont=cmunbx,
    ItalicFont=cmunti,
    BoldItalicFont=cmunbi,
]{cmunrm}
\usepackage{polyglossia}
\setdefaultlanguage{russian}
\setotherlanguage{english}


\usepackage{geometry}
\usepackage{pgfplotstable}

\geometry{
margin=2cm
}

% Создаем команду, чтобы переносить текст на новую строку внутри таблицы
\newcommand{\tcell}[2][l]{\begin{tabular}[#1]{@{}c@{}}#2\end{tabular}}

\usepackage{indentfirst}

\usepackage{arydshln}
\usepackage[fleqn]{amsmath}
\usepackage{xfrac}
\usepackage{esint}
\usepackage{amssymb}
\usepackage{mathbbol}
\usepackage[T1]{fontenc}
\usepackage{mathtools}
\usepackage{color}
\usepackage{ulem}
\usepackage{tabu}
\usepackage{multirow}
\usepackage{rotating}
\usepackage{enumitem}

\usepackage[outline]{contour}
\contourlength{1.2pt}

\usepackage{tikz}
\usepackage{graphics}
\usepackage{xcolor}

\usepackage{pgfplots}
\usepackage{pgfplotstable}

\usepackage[at]{easylist}

\DeclareMathOperator{\sign}{sign}

\newcommand{\insertTitle}[6]{
\begin{titlepage}
	\begin{center}
    	\large
		Министерство науки и высшего образования Российской Федерации
		
		Новосибирский государственный технический университет
		%\vspace{0.25cm}
		\vfill
		{\textbf #1}
		
		Лабораторная работа №#2
		\vfill
	\end{center}
	
	\begin{tabular}{ m{7em}  m{7em} }
	Факультет: & ФПМИ \\ 
	Группа: & #3 \\  
	Студент: & #4 \\
	Вариант: & #5
	\end{tabular}
	\vfill

\begin{center}
Новосибирск

#6
\end{center}
\end{titlepage}
}


\newcommand{\funcA}{ \[ u(x,t) = t \cdot x \] }
\newcommand{\funcB}{ \[ u(x,t) = t \cdot 10 \cdot x \] }
\newcommand{\funcC}{ \[ u(x,t) = t \cdot x^2 \] }
\newcommand{\funcD}{ \[ u(x,t) = t \cdot x^3 \] }
\newcommand{\funcE}{ \[ u(x,t) = t \cdot sin(x) \] }
\newcommand{\funcF}{ \[ u(x,t) = t \cdot eps(x) \] }

\newcommand{\inputTable}[1]{
\begin{center}
\noindent
\pgfplotstabletypeset[
	columns/i/.style={column name={\scriptsize coef},},
	columns/nodes/.style={column name={\scriptsize число узлов},},
	columns/iters/.style={column name={\scriptsize число итераций },},
	columns/norm/.style={column name={\scriptsize норма вектора }, column type/.add={}{|},},
	every head row/.style={before row=\hline,after row=\hline\hline}, 
	every last row/.style={after row=\hline},
	column type/.add={|}{},
	col sep=tab,
]{#1}
\end{center}
}





\usepackage[utf8]{inputenc}
\usepackage{listings}
\usepackage{color}
 
\definecolor{codegreen}{rgb}{0,0.6,0}
\definecolor{codegray}{rgb}{0.5,0.5,0.5}
\definecolor{codepurple}{rgb}{0.58,0,0.82}
\definecolor{backcolour}{rgb}{0.95,0.95,0.95}
 
\lstdefinestyle{mystyle}{
    backgroundcolor=\color{backcolour},   
    commentstyle=\color{codegreen},
    keywordstyle=\color{blue},
    numberstyle=\tiny\color{codegray},
    stringstyle=\color{codepurple},
    basicstyle=\ttfamily,
    breakatwhitespace=false,         
    breaklines=true,                 
    captionpos=b,                    
    keepspaces=true,                 
    numbers=left,                    
    numbersep=5pt,                  
    showspaces=false,                
    showstringspaces=false,
    showtabs=false,                  
    tabsize=4
}


\newcommand{\myCodeInput}[3]{
{\bf #2}
\lstinputlisting[language=#1]{#3}
}



%-------------------------------------------------------------------------------
%-------------------------------------------------------------------------------
%-------------------------------------------------------------------------------

\lstset{style=mystyle}

\begin{document}

\setlength{\abovedisplayskip}{0.1pt}
\setlength{\belowdisplayskip}{0.1pt}

%-------------------------------------------------------------------------------
\insertTitle{Уравнения математической физики}{2}{ПМ-63}{Кожекин М.В.}{5}{2019}


%-------------------------------------------------------------------------------
\section{Цель работы}
Разработать программу решения нелинейной одномерной краевой задачи методом конечных элементов. Сравнить метод простой итерации и метод Ньютона для решения данной задачи.


%-------------------------------------------------------------------------------
\section{Задание}
1.	Выполнить конечноэлементную аппроксимацию исходного уравнения в соответствии с заданием. Получить формулы для вычисления компонент матрицы   и вектора правой части   для метода простой итерации. 

2.	Реализовать программу решения нелинейной задачи методом простой итерации с учетом следующих требований:
\begin{itemize}[noitemsep]
\item язык программирования С++ или Фортран;
\item предусмотреть возможность задания неравномерных сеток по пространству и  по времени, разрывность параметров уравнения по подобластям, учет краевых условий;
\item матрицу хранить в ленточном формате, для решения СЛАУ использовать метод  -разложения;
\item предусмотреть возможность использования параметра релаксации.
\end{itemize}

3.	Выполнить линеаризацию нелинейной системы алгебраических уравнений с использованием метода Ньютона. Получить формулы для вычисления компонент линеаризованных матрицы   и вектора правой части  

4.	Реализовать программу решения нелинейной задачи методом Ньютона.

5.	Протестировать разработанные программы.

6.	Исследовать реализованные методы на различных зависимостях коэффициента от решения (или производной решения) в соответствии с заданием. На одних и тех же задачах сравнить по количеству итераций метод простой итерации и метод Ньютона. Исследовать скорость сходимости от параметра релаксации.


{\bf Вариант 5:}
Базисные функции линейные.

\[ -div(\lambda(u)grad(u)+\sigma \frac{du}{dt} = f \]



%-------------------------------------------------------------------------------
\section{Анализ}

Произведя временную аппроркимацию  по двуслойной неявной схеме исходное уравнение примет вид:

\[ -div(\lambda(u)grad(u) + \frac{\sigma}{\Delta t_s} u_s = f + \frac{\sigma}{\Delta t_s} u_{s-1} \]


В ходе конечноэлементной аппроксимации нелинейной начально-краевой задачи получается система нелинейных уравнений

\[ { \bf A(q_s)q_s = b(q_s)} \]

у которой

\[ G_{i,j} = \int_{\Omega}{\lambda(\frac{du}{dx}) grad\psi_i grad\psi_j d\Omega} = \]

\[ G_{0,0} = \sum_{k=0}^{1} \int_{\Omega}{\lambda(\frac{q_1 - q_0}{h}) \psi_k grad\psi_0 grad\psi_0 d\Omega} = \]

\[ = \sum_{k=0}^{1} \int_{\Omega}{\lambda(\frac{q_1 - q_0}{h}) \psi_k grad\psi_0 grad\psi_0 d\Omega} =\]

\[ = \frac{\lambda_0(\frac{q_1-q_0}{h}) + \lambda_1(\frac{q_1-q_0}{h})}{2h} =  G_{1,1} \]

\[ G_{0,1} = \sum_{k=0}^{1} \int_{\Omega}{\lambda(\frac{q_1 - q_0}{h}) \psi_k grad\psi_0 grad\psi_1 d\Omega} = \]

\[ = \sum_{k=0}^{1} \int_{\Omega}{\lambda(\frac{q_1 - q_0}{h}) \psi_k grad\psi_0 grad\psi_1 d\Omega} = \]

\[ = -\frac{\lambda_0(\frac{q_1-q_0}{h}) + \lambda_1(\frac{q_1-q_0}{h})}{2h} = G_{1,0}\]

%\par\noindent\rule{\textwidth}{0.4pt}
\[ G = \frac{\lambda_0(\frac{q_1-q_0}{h}) + \lambda_1(\frac{q_1-q_0}{h})}{2h}\begin{pmatrix} 1 & -1 \\ -1 & 1 \end{pmatrix} \]
\par\noindent\rule{\textwidth}{0.4pt}

\[ M_{i,j} = \frac{\sigma}{\Delta t_s}\int_{\Omega}{\psi_i\psi_jd\Omega} \]

\[ M_{0,0} = \frac{\sigma}{\Delta t_s}\int_{\Omega}{\psi_0\psi_0d\Omega} =
\frac{\sigma h}{\Delta t_s} \int_0^1{\xi^2 d\xi } =
\frac{\sigma h}{\Delta t_s} \frac{\xi^3}{3}\Bigr|_{0}^{1} = \frac{\sigma h}{3 \Delta t_s} = M_{1,1}
\]

\[ M_{0,1} = \frac{\sigma}{\Delta t_s}\int_{\Omega}{\psi_0\psi_1d\Omega} =
\frac{\sigma h}{\Delta t_s} \int_0^1{\xi(1-\xi) d\xi } =
\frac{\sigma h}{\Delta t_s} \bigl( \frac{\xi^2}{2} - \frac{\xi^3}{3} \bigr) \Bigr|_{0}^{1} = \frac{\sigma h}{6 \Delta t_s} = M_{1,0}
\]

%\par\noindent\rule{\textwidth}{0.4pt}
\[ M = \frac{\sigma h}{6 \Delta t_s}\begin{pmatrix} 2 & 1 \\ 1 & 2 \end{pmatrix} \]
\par\noindent\rule{\textwidth}{0.4pt}

\[ b_i = \int_{\Omega}{f_s\psi_i d\Omega}+\frac{1}{\Delta t_s}\int_{\Omega}{\sigma u_{q-1}^h \psi_id\Omega} \left | u_{q-1}{h} = \sum_{k=0}^1{q_{k,s-1}\psi_k} \right |
\]

\[ b_0 = fh\int_{0}^{1}{\xi d\xi}+\frac{\sigma}{\Delta t_s}\sum_{k=0}^{1} \int_{\Omega}{q_{k,q-1} \psi_k \psi_0 d\Omega}
\]

\[ =fh\frac{\xi^2}{2}\Bigr|_{0}^{1} + \frac{\sigma}{\Delta t_s}\Bigl[ q_{0,s-1}\int_{\Omega}{\psi_0 \psi_0 d\Omega} + q_{1,s-1}\int_{\Omega}{\psi_1 \psi_0 d\Omega} \Bigr]
\]

\[ =\frac{fh}{2} + \frac{\sigma}{\Delta t_s}\Bigl[ q_{0,s-1}\int_{\Omega}{\xi^2 d\xi} + q_{1,s-1}\int_{\Omega}{\xi (1-\xi) d\xi} \Bigr]
\]

\[ =\frac{fh}{2} + \frac{\sigma}{\Delta t_s} \Bigl[ q_{0,s-1}\frac{\xi^3}{3} \Bigr|_{0}^{1} + q_{1,s-1}\bigl( \frac{\xi^2}{2} - \frac{\xi^3}{3} \bigr) \Bigr|_{0}^{1} \Bigr]
\]

\[ =\frac{fh}{2} + \frac{\sigma}{\Delta t_s} \Bigl[ \frac{1}{3}q_{0,s-1} + \frac{1}{6}q_{1,s-1} \Bigr]
=\frac{fh}{2} + \frac{\sigma}{6 \Delta t_s} \Bigl[ 2q_{0,s-1} + q_{1,s-1} \Bigr]
\]

\[ b_1 = fh\int_{0}^{1}{(1-\xi) d\xi}+\frac{\sigma}{\Delta t_s}\sum_{k=0}^{1} \int_{\Omega}{q_{k,q-1} \psi_0 \psi_1 d\Omega}
\]

\[=fh\frac{\xi^2}{2}\Bigr|_{0}^{1} + \frac{\sigma}{\Delta t_s}\Bigl[ q_{0,s-1}\int_{\Omega}{\psi_0 \psi_1 d\Omega} + q_{1,s-1}\int_{\Omega}{\psi_1 \psi_1 d\Omega} \Bigr]
\]

\[=\frac{fh}{2} + \frac{\sigma}{\Delta t_s}\Bigl[ q_{0,s-1}\int_{\Omega}{\xi (1-\xi) d\xi} + q_{1,s-1}\int_{\Omega}{(1-\xi)^2 d\xi} \Bigr]
\]

\[=\frac{fh}{2} + \frac{\sigma}{\Delta t_s} \Bigl[ q_{0,s-1}\bigl( \frac{\xi^2}{2} - \frac{\xi^3}{3} \bigr) \Bigr|_{0}^{1} + q_{1,s-1}(1 - \xi)^3 \Bigr|_{0}^{1} \Bigr]
\]

\[=\frac{fh}{2} + \frac{\sigma}{\Delta t_s} \Bigl[ \frac{1}{6}q_{0,s-1} + \frac{1}{3}q_{1,s-1} \Bigr]
=\frac{fh}{2} + \frac{\sigma}{6 \Delta t_s} \Bigl[ q_{0,s-1} + 2 q_{1,s-1} \Bigr]
\]

%\par\noindent\rule{\textwidth}{0.4pt}
\[ b = \begin{pmatrix} \frac{fh}{2} + \frac{\sigma}{6 \Delta t_s} \Bigl[ 2q_{0,s-1} + q_{1,s-1} \Bigr] \\ 
							\frac{fh}{2} + \frac{\sigma}{6 \Delta t_s} \Bigl[ q_{0,s-1} + 2 q_{1,s-1} \Bigr]
		\end{pmatrix} \]
\par\noindent\rule{\textwidth}{0.4pt}


%-------------------------------------------------------------------------------
\section{Исследования}
\funcA{}
\inputTable{file_u0.txt}
\funcB{}
\inputTable{file_u1.txt}
\funcC{}
\inputTable{file_u2.txt}
\funcD{}
\inputTable{file_u3.txt}
\funcE{}
\inputTable{file_u4.txt}
\funcF{}
\inputTable{file_u5.txt}






%-------------------------------------------------------------------------------
\section{Выводы}



%-------------------------------------------------------------------------------
%\section{Исходный код программы}
%\myCodeInput{c++}{head.h}{../head.h}
%
%\myCodeInput{c++}{grid.h}{../grid.h}
%\myCodeInput{c++}{grid.cpp}{../grid.cpp}
%
%\myCodeInput{c++}{fem.h}{../fem.h}
%\myCodeInput{c++}{fem.cpp}{../fem.cpp}
%
%\myCodeInput{c++}{solver.h}{../solver.h}
%\myCodeInput{c++}{solver.cpp}{../solver.cpp}
%
%\myCodeInput{c++}{main.cpp}{../main.cpp}

\end{document}

